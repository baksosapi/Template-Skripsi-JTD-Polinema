\chapter{Tinjauan Pustaka}
\label{chap: pustaka}
\section{Penelitian terdahulu}
\label{sec:penelitianterdahulu}

Bagian ini menyajikan tentang semua yang dijadikan latar belakang dari pengambilan topik skripsi.
Termasuk juga masalah-masalah yang akan dihadapi untuk membuatnya, termasuk kurangnya kemampuan penguasaan \LaTeX{} sehingga template ini dibuat dengan mengandalkan berbagai contoh yang tersebar di dunia maya, yang digabung-gabung menjadi satu jua.
Bagian lain juga akan dilengkapi, untuk sementara diisi dengan lorem ipsum versi bahasa inggris.

\dtext{5-10}

\section{Kajian Teori}
\label{sec:teori}
Kajian Teori (sesuaikan judul sub bab ini dengan isi kajian teori / pustaka yang relevan)

\dtext{6}

\section{Hipotesis}
\label{sec:hipotesis}
Jika terdapat hipotesis , maka jelaskan di sini

\dtext{8}