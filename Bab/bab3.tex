\chapter{Metode Penelitian}
\label{chap: metode}

Jenis Penelitian (bagian ini menentukan prosedur penelitian berikutnya).

\section{Jenis penelitian adalah Kuantitatif}
\label{chap: kuantitatif}
Jika jenis penelitian kuantitatif (yang tidak merencanakan alat), maka:

\begin{itemize}
    \item 3.1 Jenis penelitian adalah Kuantitatif
    \item 3.2 Rancangan penelitian dapat berbentuk eksperimen atau non eksperimen
    \item 3.3 Sampel dan Teknik Pengambilan Sampel
    \item 3.4 Data dan Teknik Pengumpulan Data
    \item 3.5 Teknik Analisis/Pengolahan DataJenis penelitian adalah Kuantitatif
\end{itemize}

Jika jenis penelitian Pembuatan/Pengembangan (Alat, Sistem, Aplikasi), maka:
\begin{itemize}
    \item 3.1 Jenis Penelitian adalah Pembuatan/Pengembangan
    \item 3.2 Rancangan penelitian berbentuk Rancang Bangun
    \item 3.3 Perancangan (Alat, Sistem, Aplikasi)
    \item 3.4 Penyiapan Bahan dan Alat
    \item 3.5 Penentuan prosedur dan parameter
\end{itemize}

\section{Rancangan penelitian dapat berbentuk eksperimen atau non eksperimen}
\label{sec:rancangan}
Kajian Teori (sesuaikan judul sub bab ini dengan isi kajian teori / pustaka yang relevan)

\dtext{6}

\section{Sampel dan Teknik Pengambilan Sampel}
\label{sec:sampel}
Jika terdapat hipotesis , maka jelaskan di sini

\dtext{8}


\section{Teknik Analisis/Pengolahan Data}
\label{sec:olahdata}
Jika terdapat hipotesis , maka jelaskan di sini

\dtext{8}
